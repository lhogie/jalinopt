%%This is a very basic article template.
%%There is just one section and two subsections.
\documentclass{article}
/Users/lhogie/luc.tex
\title{Linear Programming with \jalinopt}
\author{Luc Hogie (CNRS/INRIA/UNS)}
\begin{document}
\maketitle



\section{Introduction}

\subsection{The idea}

Many mathematical and engineering problems can be expressed as linear programs, and doing so facilitates their resolution. Indeed it is generally more convenient to transform a domain-specific problem into a linear-optimizable one (that can be solved by any solver) than writing a complex domain-specific algorithm.

\jalinopt is a Java toolkit for building and solving linear programs.
It consists of a straightforward
object-oriented model for linear programs,
as well as a bridge to most common solvers, including GLPK and CPLEX.

Altought \jalinopt  is inspired by Mascopt and JavaILP, it provides a significantly different OO model and
an utterly different approach to connecting to the solver. In particular this approach offers better portability
and the possibility to connect (via SSH) to solvers on remote computers. 

The working process of \jalinopt is the following:
\begin{enumerate}
  \item the user defines a linear program as a Java object;
  \item \grph translates this object into a plain CPLEX LP text;
  \item this LP text is piped via standard input to a solver that is executed in another process;
  \item on completion of the resolution, the standard output of the solver is piped back to \grph;
  \item \grph parses this output text and builds a result in the form of a Java object;
  \item this result object is given to the user, who is invited to extract the information he needs to
  solve his graph problem. 
\end{enumerate}

Only items 1 and 6 require the intervention of the programmer.


\subsection{JNI vs. piping}

Weirdly enough, there does not exist Java implementation of linear solvers. At most one can find implementations
of the simplex method. A Java linear solver would be worth having because this would solve portability issues.

As a consequence, Java programs dealing with linear optimization need to interact with native node. Such interaction
can be achieve in different ways. Let us consider Java Native Invocation (JNI) and piping.

JNI is a feature of the Java Virtual Machine (JVM) that allows  Java programs to run native code in the same process
at the JVM itself. To do so, the JVM first dynamically loads a library file and connect Java methods to native functions.
JNI is the solution proposed by most linear solvers by providing the Java developer with the adequate Java JNI-enabled interfaces. 
 Unfortunately, in practise 
JNI's  dynamic linkage of natively compiled libraries is a problem:  the great variety of processors, operating systems
types and versions imposes that JNI-based Java programs must embed all possible JNI-loadable libraries, which proves
impracticable.
 
 Piping is an inter-process communication scheme most known in the UNIX family of operation systems.
 A pipe is an unidirectional binary communication channel. The communication between the source process and the destination one
 all happens in memory using sophisticated memory management algorithms, making the  transfer of data highly efficient.
 Although Java does not provide standard API for inter-process pipes, they can be found on all operation systems and hence are
 a very portable solution. 

 Pipe-based communication imposes that involved processed agree on a coding of the data exchanged. This imposes that
 the calling process:
 \begin{itemize}
  \item transforms the data to be transferred to a sequence of bytes that the native process can understand;
  \item parses the data coming from the native process. 
\end{itemize}
 
Speaking about performance, whether one resort to JNI or piping, the same amount of data must be transferred to
the native process. In both case, an adaptation of the data must be done: while JNI must convert types and encoding in both direction, piping-based solutions
must convert bytes to native data and vice-versa.
 
 
 
\subsection{State of the Art}

Mascopt is a graph problems optimization library. Among numerous features, it comes with  a LP manipulation API.
Calls to solvers are done through JNI. 

JavaILP provides a clean object-oriented model of linear problems and features bridges to
many solvers. Its model has an unnecessary complex representation of variable types, and its hash-table based implementation may pose problems in the cas of large LPs.
Also, just like Mascopt, calls to solvers are done through JNI. 




\section{Model for linear programs}

This section presents the API for generating linear programs.
Please note that this API is not meant to be easy to read/write by humans (this requirement is perfectly met
by the standard form of LPs). Instead it is meant to be programmatically handy, making the translation of  domain-specific problems
into linear ones as natural as possible.


\subsection{Creating a new problem}


\begin{lstlisting}
LP lp = new LP();
\end{lstlisting}

\subsection{Setting the objective}

\begin{lstlisting}
LinearExpression objective = lp.getObjective();
objective.addTerm(coefficient, variable);
\end{lstlisting}


\subsection{Setting the type of optimization}

\begin{lstlisting}
lp.setOptimizationType("min");
\end{lstlisting}


\subsection{Adding a new constraint}

A constraint is a triplet $lhs, operation, scalar$. For example:
$$
	4a + b + 2c > 10
$$



\begin{lstlisting}
Constraint constraint = lp.addConstraint();
LinearExpression lhs = constraint.getLeftHandSide();
lhs.addTerm(4, "a");
lhs.addTerm(1, "b");
lhs.addTerm(2, "c");
constraint.setOperator(">");
constraint.setRightHandSide(10);
\end{lstlisting}

 \jalinopt accepts this shorter form for the definition of constaints:



\begin{lstlisting}
Constraint constraint = lp.addConstraint(">", 10);
constraint.addTerm(4, "a");
constraint.addTerm(1, "b");
constraint.addTerm(2, "c");
\end{lstlisting}


\subsection{Defining the type of the variables}


\section{Solving a problem}


The resolution of a problem gives the optimal value for the objective function, as well as the value of each of the variables
when the objective is optimized.


\subsection{Using the default solver}


Once a problem is defined, it can be solved. To do so, one need to 

\begin{lstlisting}
Result r = lp.solve();
\end{lstlisting}

The \code{Result} object then encapsulates the value of the optimized objective as well as the value of
every variable which was used to reach the objective. 

The default solver is defined by iteratively looking for the following components, and by using the first found:

\begin{enumerate}
  \item local installation of CPLEX;
  \item an installation of CPLEX on any of the SSH-accessible hosts defined in the \texttt{~/.jalinopt/hosts} file;
  \item the Java Apache LP Solver (but this one does not support ILP).
\end{enumerate}

\jalinopt first checks if CPLEX is installed on the local host. To do so
it looks at the \texttt{PATH} environment variable. If the CPLEX executable cannot be found, \jalinopt considers the 
\texttt{LPSolver.CPLEX\_HOST} variable. If this variable is set to null

\subsection{Forcing the solver}

\begin{lstlisting}
Result r = lp.solve(new LocalCplex());
\end{lstlisting}

Available servers include:
\begin{description}
  \item[jalinopt.ApacheSolver] the solver that comes with the Apache Commons optimization toolkit;
  \item[jalinopt.cplex.LocalCPLEX] uses the cplex application, that is installed on the local computer; the \texttt{cplex} command
  must be in the system PATH;
  \item[jalinopt.cplex.CPLEX\_SSH] uses the cplex application installed on another computer; the SSH environment should be
  configured to as the user can connect to the remote host without having to type his password.
\end{description}


\subsection{On a remote computer}

This feature assumes than SSH is properly configured, so that the user does not need to provide a password when
connecting to the remote computer. It also assumes that the solver executable is located in the PATH on this
remote computer.

For the moment, only CPLEX can be invoked in a remote fashion. To process, one need to invoke the \code{CPLEX\_SSH} solver. This solver
needs information on which host to connect to, and the username on this host. This information is passed to the solver
by using a \code{SSHConnectionInfo} object. This object needs a hostname and a user name. If the user name is omitted,
\jalinopt considers the local username.

The following example solve the given LP using CPLEX on the host \texttt{musclotte}:
\begin{lstlisting}
SSHConnectionInfo sshInfo = new SSHConnectionInfo("musclotte");
Result r = lp.solve(new CPLEX_SSH());
\end{lstlisting}

The following shortcut is also accepted:

\begin{lstlisting}
Result r = lp.solve(new CPLEX_SSH("musclotte")));
\end{lstlisting}



\section{Example}

Let us consider the following LP.
\begin{lstlisting}
Minimize
	obj: 3x + 2y
Subject To
	4x + y < 15
	5y > 10
General
	y 
End
\end{lstlisting}

This LP can be programmatically modelled by the following Java code:


\begin{lstlisting}
	LP lp = new LP();
	lp.setOptimizationType(OptimizationType.MIN);
	
	// define the objective
	LinearExpression objective = lp.getObjective();
	objective.addTerm(3, "x");
	objective.addTerm(2, "y");

	// 4x + 1y < 15
	Constraint c1 = lp.addConstraint("<", 15);
	c1.getLeftHandSide().addTerm(4, "x");
	c1.getLeftHandSide().addTerm(1, "y");

	// another way to formulate a constraint (5y > 10)
	Constraint c2 = lp.addConstraint(">", 10);
	c2.getLeftHandSide().addTerm(5, "y");

	// define the type for variable
	lp.getVariableByName("y").setType(TYPE.INTEGER);
\end{lstlisting}




\end{document}
